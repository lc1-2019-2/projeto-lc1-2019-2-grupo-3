\documentclass[12pt]{article}

\usepackage[utf8]{inputenc}
\usepackage[brazil]{babel}
\usepackage{amsmath,amssymb}
\usepackage{pdfsync}
\usepackage[all]{xy}
\usepackage{color}
\usepackage[T1]{fontenc}

\title{{\large Universidade de Brasília \\ Instituto de Ciências Exatas \\
Departamento de Ciência da Computação} \\[1cm]
117366 - Lógica Computacional Turma A\\[.5cm]
Relatório sobre {\bf Provas de conjecturas do Merge Sort e Radix Sort}}
\author{Alexandre Mitsuru Kaihara - 18/0029690 \\
        Christian Luis Marcondes Costa - 15/0153538}
\date{\today}


\begin{document} 
	\maketitle
	
	\section{Introdução}
	    Na computação, ao desenvolver algoritmos, muitas vezes depara-se com o problema de ordenação de listas. Para ordenar uma lista qualquer, temos à disposição uma grande quantidade de algoritmos, entre eles o merge sort e o radix sort. 
	    
	    O merge sort funciona da seguinte forma: tendo uma lista qualquer, é feita a divisão desta lista em elementos unitários, em seguida pegamos esses elementos unitários e formamos uma sublista fazendo a combinação ordenada desses elementos 2 à 2 (se possível), em seguida fazemos a ordenação entre cada uma dessas sublistas, pegando sempre o menor elemento (ou o maior para o caso de ordenação decrescente) entre as duas cabeças dessa lista e adicionando esse elemento na cauda de uma nova sublista. Quanto chegar ao ponto que temos apenas uma sublista, esta lista será uma permutação ordenada da lista inicial.
	    
	    O radix sort é implementado da seguinte maneira: considere uma lista não ordenada de inteiros. Para ordenar essa lista é observado o dígito menos significativo de cada um desses números, e a ordenação é feita a partir desse valor. Em seguida pegamos o digito subsequente e realizamos a mesma operação ordenação, até que seja alcançado o dígito mais significativo do maior número da lista. Ao completar essa sequência de ordenações, obteremos uma lista ordenada da lista inicial. Entretanto, é necessário ter um algoritmo auxiliar para fazer a ordenação da lista em cada um desses passos. 
	    
	    Para que o algoritmo seja estável, ou seja, que não haja a troca da ordem original entre elementos iguais, é necessário um algoritmo estável para ordenar os dígitos em cada um dos passos do radix sort. Para resolver este problema é utilizado o merge sort nos passos auxiliares.
	    
	    O objetivo deste projeto é provar a corretude dos algoritmos de merge sort e radix sort utilizando técnicas dedutivas da lógica de predicados por meio do assistente de demonstração PVS. Nos capítulos abaixo é possível verificar a intuição por trás de cada prova.
	    
    \section{Questão 1: Merge sort gera uma permutação da lista L}
        
        O objetivo desta questão é provar que para toda lista L, o mergesort de L é uma permutação de L. Para isso, faremos indução no tamanho da lista L.
        
        Para o caso em que a lista tenha tamanho menor ou igual a um é trivial de se provar, pois pela definição, o merge sort de uma lista desse tamanho retorna a própria lista. Portanto, as ocorrências de duas listas iguais são as mesmas.
        Para os demais casos, partimos da seguinte fórmula:
        
        \begin{verbatim}
        occurrences(merge_sort(l))(x) = occurrences(merge_sort(l))(x)
        \end{verbatim}
        
        O intuito é alcançar a igualdade de que ocorrências de x em L são as mesmas das ocorrências de x do merge sort de L.
        Dado que o merge sort é um merge do prefixo e sufixo da lista L, obtém-se:
         
        \begin{verbatim}
        occurrences(merge(merge_sort(prefix(x ,floor(length(x)/2)))),
        merge_sort(suffix(x ,floor(length(x)/2)))))(x) 
        = occurrences(merge_sort(l))(x)
        \end{verbatim}
        
        Com o lema de que as ocorrências de um merge de duas listas quaisquer é a soma das ocorrências da primeira com as ocorrências da segunda, substitui-se na fórmula.
        
        \begin{verbatim}
        occurrences(merge_sort(prefix(x ,floor(length(x)/2))))(x) +
        occurrences(merge_sort(suffix(x ,floor(length(x)/2))))(x) 
        = occurrences(merge_sort(l))(x)
        \end{verbatim}
        
        A partir da nossa hipótese de indução, cuja definição garante que para toda lista y de tamanho menor que a lista x implica que as ocorrências do merge sort y são as mesmas que na lista y, assim, para a lista do prefixo de x e sufixo de x tem-se:
        
        \begin{verbatim}
        occurrences(merge_sort(prefix(x ,floor(length(x)/2))))(x) =
        occurrences(prefix(x ,floor(length(x)/2)))(x) 
        \end{verbatim}
        
        \begin{verbatim}
        occurrences(merge_sort(suffix(x ,floor(length(x)/2))))(x) =
        occurrences(suffix(x ,floor(length(x)/2)))(x) 
        \end{verbatim}
        
        A partir dessas hipóteses de indução chegamos em:
        
        \begin{verbatim}
        occurrences(prefix(x ,floor(length(x)/2)))(x) + 
        occurrences(suffix(x ,floor(length(x)/2)))(x) =
        occurrences(merge_sort(l))(x)
        \end{verbatim}
        
        Por fim, utiliza-se o lema de que a soma de duas ocorrências é o mesmo da ocorrência do apêndice das duas listas, para enfim aplicar outro lema que diz que o apêndice do prefixo e sufixo da lista L é a própria lista L, resultando em:
        
         \begin{verbatim}
        occurrences(l)(x) = occurrences(merge_sort(l))(x)
        \end{verbatim}
        
        Enfim, prova-se que o merge sort gera uma permutação da lista L.
    
    \section{Questão 2: Radix sort gera uma permutação da lista L}
    
        Nesta questão, o objetivo é provar que para toda lista L de um tipo T qualquer, temos que radixsort de L retorna uma permutação de L. Entretanto, sabe-se que o radixsort de L nada mais é do que um mergesort da pré-ordem << de outro mergesort de pré-ordem <= da lista L. Para provar, é necessário considerar que o mergesort de L gera uma permutação da lista L - provado na questão 1. Também, utilizando-se desse lema, é possível a aplicação para as duas possíveis pré-ordens << e <= - sendo duas ordenações distintas quaisquer. 
        
        A partir da definição de permutação, temos que para todo item x de uma lista L qualquer temos a mesma quantidade de ocorrências deste mesmo item x em outra lista L'. A partir disso, pode-se afirmar que as ocorrências de x em mergesort de L na pré-ordem <= são iguais as ocorrências de x na lista L (questão 1). Obtém-se a seguinte formula:
        
        \begin{verbatim}
        occurrences(merge_sort[T, <=](l))(x) 
        = occurrences(l)(x)
        \end{verbatim}
        
        A partir do enunciado do problema e dado que a permutação é a ocorrência dos mesmos elementos de duas listas e a definição de radix sort, obtém-se a formula:
        
        \begin{verbatim}
        occurrences(merge_sort[T,<<](merge_sort[T, <=](l)))(x) 
        = occurrences(merge_sort[T, <=](l))(x)
        \end{verbatim}
        
        Ao substituir a primeira fórmula na segunda obtém-se:
        
        \begin{verbatim}
        occurrences(merge_sort[T,<<](merge_sort[T, <=](l)))(x) 
        = occurrences(l)(x)
        \end{verbatim}
        
        Dado que a definição de radix sort é um merge sort de um merge sort da lista L com as pré-ordem << e <=, respectivamente, conclui-se a prova.
        
        \begin{verbatim}
        occurrences(radixsort(l)))(x) = occurrences(l)(x)
        \end{verbatim}
        
    \section{Questão 3: Radix sort ordena}
    
        Aqui é necessário provar que radixsort está ordenado na ordem lexicográfica, ou seja, uma ordenação similar à do dicionário. Para iniciar essa prova, usaremos dois lemas: o lema de que mergesort é ordenado na ordenação << e o lema de que uma lista ordenada implica que os elementos seguem uma função monótona, ou seja, que todos os elementos estão seguindo uma mesma ordem, seja ela crescente ou decrescente.
        
        A intuição por trás do uso dos lemas é deixar o antecedente com a mesma forma da fórmula do consequente. Com isso, temos um sequente da forma A implica B e outro sequente A no antecedente, podemos realizar uma simplificação e obter o sequente A e o sequente B apenas, eliminando a implicação.
        
        O próximo passo realizado foi expandir a definição de radixsort está ordenado. Para fins de organização e legibilidade do relatório, foram substituidas as ocorrências de mergesort[T, <<](mergesort[T, <=](l)) por radixsort(l) nas fórmulas.
        
        \begin{verbatim}
        FORALL (
        j: below[length(radixsort(l))],
        i: below[j]): 
            nth(radixsort(l), i) << nth(radixsort(l), j)
        |---------
        FORALL (k: below[length(radixsort(l))]):
        k <= length(radixsort(l)) - 2 
        => lex(nth(radixsort(l), k), nth(radixsort(l), 1 + k))
        \end{verbatim}
        
        Como tanto o antecedente e o consequente são fórmulas de ordenação, faremos a expansão das fórmulas até que se chegue em algo comum no consequente e antecedente. Para isso, na fórmula do consequente foi assumido um k qualquer, e para instanciado para j o valor de k + 1 e para i o valor de k, por fim expandimos a definição de lex. Lex nos diz que:
        \begin{verbatim}
        (x << y AND NOT (y << x))  
         OR 
        (( x << y AND y << x) AND x <= y )
        \end{verbatim}
        
        Em nossa prova, x e y são, respectivamente:
        \begin{verbatim}
        (nth(radixsort(l)), k + 1)
        (nth(radixsort(l)), k)
        \end{verbatim}
        
        Queremos dividir essa prova em dois casos, um no qual x << y é verdadeiro e outro no qual x << y é falso. Com isso, teremos uma ramificação da árvore com x << y no antecedente e outra ramificação x << y no consequente.
        
        A ramificação com x << y no consequente pode ser resolvida com uma manipulação de fórmulas. Considere x << y como A e y << x como A\textsuperscript{-1}, além disso, considere y <= x como B. Em nosso consequente, temos:
        \\
        
        NOT (A\textsuperscript{-1}) OR A OR ((A\textsuperscript{-1}) AND NOT A) OR       (A\textsuperscript{-1} AND A AND B) \\
        
        Manipulando esta fórmula com as propriedades distributivas temos:
        \\
        
        ((NOT (A\textsuperscript{-1}) OR (A\textsuperscript{-1}) OR A) AND (NOT (A\textsuperscript{-1}) OR A OR NOT A)) OR (A\textsuperscript{-1} AND A AND B) \\
        
        O que nos gera uma tautologia no consequente. Passando esta tautologia para o antecedente, temos um absurdo que nos leva ao axioma e conclui este ramo da prova.
        
        Já no outro ramo da árvore temos x << y no antecedente, realizando manipulações e simplificações de fórmula conseguimos obter B no consequente como única fórmula. Nosso objetivo é conseguir fazer que B apareça no antecedente e para isso aconteça usaremos o lema que diz que mergesort é conservativo, pois esse lema implica em uma fórmula que pode simplificar B. Além disso, temos x << y e y << x na esquerda dessa implicação do lema e temos essas mesmas fórmulas no antecedente. Simplificando o antecedente, obtemos a fórmula abaixo.
        
        \begin{verbatim}
        EXISTS (i, j: below[length(merge_sort[T, <=](l))]):
          i < j AND
           nth(merge_sort(merge_sort[T, <=](l)), k) =
            nth(merge_sort[T, <=](l), i)
            AND
            nth(merge_sort(merge_sort[T, <=](l)), 1 + k) =
             nth(merge_sort[T, <=](l), j)
        \end{verbatim}


        Tomando i e j como valores que satisfazem a fórmula, temos as igualdade que necessitamos para simplificar a fórmula B. Reescrevendo B temos:
        
        \begin{verbatim}
        nth(merge_sort[T, <=](l), i) <= 
        nth(merge_sort[T, <=](l), j)
        \end{verbatim}
        
        Nesta parte da prova, usaremos o lema que diz que lista ordenada implica em função monótona e o lema que diz que mergesort é ordenado. Com a fusão desses dois lemas, utilizando a ordem <= em mergesort, temos apenas a função monótona, da seguinte maneira.
        
        \begin{verbatim}
        FORALL (
        j: below[length(merge_sort[T, <=](l))], 
        i: below[j]):
        nth(merge_sort[T, <=](l), i) <= 
        nth(merge_sort[T, <=](l), j)
        \end{verbatim}
        
        Que é uma generalização de B no consequente. Tomando i e j como valores quaisquer temos um axioma e assim concluímos a prova da questão 3.
        
        
 \end{document}

%%% Local Variables:
%%% mode: latex
%%% TeX-master: t
%%% End: